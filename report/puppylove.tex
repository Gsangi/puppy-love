%
% LaTeX template for prepartion of submissions to SIGTBD'16
%
% Requires temporary version of sigplanconf style file provided on
% SIGTBD'16 web site.
%
\documentclass[sigtbd]{sigtbd-style}
% \documentclass[SIGTBD-cameraready]{sigplanconf-SIGTBD16}
%
% the following standard packages may be helpful, but are not required
%
\usepackage{courier}            % standard fixed width font
\usepackage[scaled]{helvet} % see www.ctan.org/get/macros/latex/required/psnfss/psnfss2e.pdf
\usepackage{url}                  % format URLs
\usepackage{listings}          % format code
\usepackage{epigraph}
\usepackage{enumitem}      % adjust spacing in enums
\usepackage[colorlinks=true,allcolors=blue,breaklinks,draft=false]{hyperref}   % hyperlinks, including DOIs and URLs in bibliography
% known bug: http://tex.stackexchange.com/questions/1522/pdfendlink-ended-up-in-different-nesting-level-than-pdfstartlink
\newcommand{\doi}[1]{doi:~\href{http://dx.doi.org/#1}{\Hurl{#1}}}   % print a hyperlinked DOI

% Comments
\newcommand{\va}[1]{\textcolor{red}{{\em VA:} #1}}
\newcommand{\sss}[1]{\textcolor{red}{{\em SS:} #1}}
\newcommand{\mv}[1]{\textcolor{red}{{\em MV:} #1}}

\newcommand{\name}{\textit{Puppy Love}}

\begin{document}

\title{\name{}: Cryptographically secure anonymous couple matching}

%
% any author declaration will be ignored  when using 'SIGTBD' option (for double blind review)
%

\authorinfo{Saksham Sharma}
{\makebox{Computer Science and Technology} \\
\makebox{Indian Institute of Technology Kanpur}  \\
{sakshams@cse.iitk.ac.in}}

\maketitle
\begin{abstract}
  Anonymous and secure dating algorithms are not only hard to define,
  but apparently tough to realize as well. Tinder et all have
  succeeded in making this a mainstream application, and yet, there is
  scope left in improving the anonymity of such platforms to provide
  a guarantee to the users that their choices shall not be made
  public, nor known to the administrators.

  We designed an algorithm for this problem which provides strong
  guarantees about privacy of users, and implemented it inside our
  campus, catering to almost 2000 users. The algorithm is designed to
  be easily scalable, and was a success in the university setting.
\end{abstract}

\section{Introduction}

The queerly named \title{} platform has been running in IIT Kanpur
since 2014, meant to help shy nerds meet their crush. The platform
opens one week prior to the Valentine's Day every year, and lets people
choose up to 4 of their crushes. At the stroke of the midnight hour on
14th February, people who happen to like each other are informed about
the same. If your \textit{love} was unrequited (the other person
didn't like you), you will not get to know. More importantly, if you
did not like the other person, you would not know if that person liked
you or not. In addition, people can see the count of people who have
selected their names without finding out who they are.

Since a campus is often a close knit circle, users are apprehensive
about revealing their crushes to others, including the administrators
(who are students themselves), lest it leak out. That makes it
necessary to design a technique to ensure that users are guaranteed
that obtaining their choices is either impossible or computationally
infeasible for anyone but them.

We developed a new version of the platform which provides strong
privacy guarantees without compromising on performance or features.

\section{Requirements from the algorithm}
The following discussion refers to $Alice$ and $Bob$. All
arguments apply to each of them.\\
The security goals are the following:
\begin{enumerate}
\item $Alice$ should not find out the choice of person $Bob$ if $Alice$
  did not like $Bob$.
\item It should not be possible for $Alice$ to cheat and obtain
  whether $Alice$ matched with the $Bob$, without this
  information being made available for $Bob$. In other words, the
  algorithm should be fair, and ideally symmetric.
\item The administrator should not be able to obtain the choices of
  a person, or brute force the data in any way to deduce the likes
  of a person.
\item The administrator should not find out which people matched, but
  knowing the count of matches is reasonable.
\item The user should be able to verify that the frontend is not
  sending any such information which could reveal their
  choices. Thus, the privacy guarantees should hold whatever code is
  running on the backend.
\item The counting technique should function independently without
  revealing the identities of the users.
\item Security features should be practical to implement in a web
  browser and should be completely transparent to the user.
\item It can be assumed that the server does not collude with
  another involved party. Thus, the server is assumed to be
  honest-but-curious and will not disrupt the service intentionally.
\end{enumerate}
Some other practical requirements:
\begin{enumerate}
\item The algorithm should have no false positives or false negatives.
\item It should ideally involve only $O(1)$ computation by each
  involved party.
\item There should ideally be no to-and-fro communication between the
  involved parties.
\end{enumerate}

\section{Related work}
A similar problem was proposed and tackled in \cite{Senpai:1} where
the proposed various algorithms, including one which does present a
secure protocol. The drawback of such an algorithm is that it requires
all involved parties to take part in $O(n)$ computation, where $n$ is
the number of other parties.

Our algorithm is completely different from this work, but it is
interesting to look at the mentioned paper nonetheless.

\bibliographystyle{abbrvnat}

\section{Initial approaches}
We went through a lot of approaches for this problem, before finally
arriving at the presented approach, which, although extremely simple,
is not obvious at first.

We initially modelled the problem as computing the AND of $O(n^2)$ bits,
disregarding the requirement of $O(1)$ computations per person.

\subsection{Homomorphic computation}

Our initial attempts at homomorphic were foiled by the fact that
completely homomorphic systems are impractical.

In contrast, partially homomorphic functions, although practical, do
not provide any function which can be used for this purpose. The
available computations are XOR, multiply, addition over 1 bit et
all. Composing any two of these functions provides enough information
to both parties to violate privacy concerns.

\subsection{Yao's two party computation}
Yao presented a secure protocol to compute any arbitrary function
(which can be run using logic gates) across two parties in a shared
manner. The protocol was not originally intended to be fair though,
and one party could cheat in some respect.

We describe the protocol here:
\begin{itemize}
\item Alice party creates a truth table for an AND gate, with columns
  A (her choice), B (Bob's choice) and C (result).
\item She selects 6 random values $a_0, a_1, b_0, b_1, c_0, c_1$. She
  then shuffles the table, and encrypts $((A = i) \& (B = j)) = Ck$
  with the value $a_i-b_j$. She then shares this table with Bob.
\item If Alice's boolean choice is $choice$, she sends $a_{choice}$ to
  Bob as well.
\item Bob obtains $b_{bob's choice}$ from Alice using Oblivious
  transfer, such that Alice will not know which bit value was obtained
  by Bob.
\item Bob tries decrypting the table with the key $a_{alice}-b_{bob}$
  and finds out what value he obtained. He can then look up whether
  they matched or not.
\end{itemize}

We attempted to make the protocol fair by using the central server,
who would then match both parties and declare their result to them.

A clever hack around this was found on noticing the asymmetry in the
algorithm. Alice could forge the truth table to return a $mismatch$ if
Bob likes Alice, and a $match$ otherwise. A false positive can be
borne easily, but if Bob did like Alice, he would not know that Alice
has cheated and has found out about his liking. This is certainly
unacceptable.

\subsection{Public key encrypted messages}
This involves the following step:
\begin{itemize}
\item All parties get to send 4 values to the server.
\item If Alice likes Bob, she will send a message (string containing
  their roll numbers) encrypted with Bob's
  public key to the server.
\item Bob will encrypt the same message with his own public key and
  store it on the server.
\item The person to use the other's public key is chosen using
  lexicographic comparison.
\item The server will detect duplicates and declare matches.
\end{itemize}

This is a very simple algorithm, but fails on a computationally
feasible brute force attack. The server can easily try out all
possible messages which Alice could have sent, and stop when it
produces a value which is the same as the one sent by Alice.

\subsection{An O(n) algorithm using duplicates}
We observed that instead of trying to compute an AND, it may be
interesting to instead use the duplicate detection in another way.

An algorithm for this is quite simple:
\begin{itemize}
\item Both parties establish an encrypted channel using their public
  keys.
\item They both exchange a random token over this channel. Let the
  tokens be $A, B$.
\item Both of them declare a value $hash(A+B)$ to the server, who
  verifies that the values are the same.
\item Each party gets to send 4 new values to the server.
\item If Alice likes Bob, she will send the pre-decided $A$ to the
  server. Otherwise she will send a random value.
\item Server computes $hash(A_{received}-B_{received})$ and matches it
  with the values received earlier. If they are the same, it declares
  a match.
\end{itemize}

Although this algorithm has no obvious security concerns, it involves
a lot of computation, infact $O(n)$ computation per person. Our tests
showed a huge performance bottleneck on modern browsers while using
Stanford's $sjcl$ library for crypto-protocol implementations.

The platform originally launched with this algorithm, but the
performance issues on scaling to $O(1000)$ people made it practically
impossible to run.

\section{The Puppy Love algorithm}
The final algorithm was inspired by the previous $O(n)$ algorithm,
wherein it was identified that it is essential to come up with a
message which only the involved parties can create, and no one
else. This is the only step which was causing a $O(n)$ bottleneck.

Apart from that, some other requirements (counting the number of
people who like you, stopping the server from knowing the matched
people) required some additions to this algorithm which are described
later.

\subsection{Matching}
\begin{itemize}
\item There is a known generator field $g$.
\item Let the public key of Alice be $g^a$ and her private key be
  $a$. The same is true for Bob's $g^b$ and $b$.
\item The value $g^{ab}$ cannot be computed by anyone other than Alice
  and Bob, since it is the Discrete Logarithm problem, which is known
  to not lie in PTime.
\item Both parties get to send 4 values to the server.
\item If Alice likes Bob, she will send $g^{ab}$ to
  the server. The same applies to Bob.
\item The server declares a match if it detects a duplicate value.
\end{itemize}

Some remarks:
\begin{enumerate}
\item The algorithm itself is quite simple, and does not present any
  apparent privacy concerns. The only issue (to be resolved later) is
  that the server will know which two people have matched.
\item It is extremely fast to implement, and running times are ideal
  for browsers, since it involves only 4 maximum mathematical
  computations.
\item Values received by the server appear completely random to it,
  and cannot be brute forced.
\item The value of $g^{ab}$ is very easily obtained if the
  Public-Private keypairs are ElGamal, as was in our case. This value
  is simply the Diffie Hellman value computed in Elliptic Curve
  Cryptography, and Stanford's $sjcl$ library provides a direct
  function for this computation.
\end{enumerate}

\subsection{Implementing the counting of people who like you}
We opted for a very simple solution to this, albeit it has a minor scope
for inconsistencies.

When Alice sends her preferences, her client also encrypts a
random string with the public key of Bob (if Bob is among her
likes). This value is stored on the server in a list ordered by
timestamps. Other clients retrieve all new entries in this list
everytime they log back in, and try decrypting all received
values. The client can detect that some of the messages were encrypted
with the user's public key, but cannot detect the sender of the
message.

\subsection{Ensuring that the server does not know the matches}
This was not implemented during 2017's deployment of the platform, but
should work fine regardless.
\begin{itemize}
\item Before sending the final chosen values to the server, the client
  obtains the blind signatures of the values it plans to send.
\item The blind signatures, as well as the actual values which are
  being sent will not be known to the server.
\item The client now switches IP (in our case, sends the packet
  through the NAT of the university, re-entering the university
  network using a reverse proxy, thus completely anonymizing the
  identity of the sender), and sends its 4 tokens (with
  signature) to the server without attaching its login cookie with
  them. The server can only verify that the received values are
  correctly signed.
\item The server declares all the matched tokens in the end, and
  clients can check if their token is among this list of matched
  tokens.
\end{itemize}

\subsection{Miscellaneous security concerns}
There are some other concerns which are discussed here:
\begin{itemize}
\item IP switching is unreliable. The user may not be expected to
  undertake this arduous task of accessing the platform behind a
  VPN/Tor for the step of sending the tokens. Thankfully, our
  university setup had a network which is known to be not in the
  control of the students, and thus the packets may use the NAT to
  ensure their identities are safe.
\item It is possible for the server to display fake public keys for
  users. This surely is a concern, but can be easily verified. The
  server does not ask for cookies while serving public keys, and thus,
  it can be easily verified (by a concerned user) whether a different
  public key is being server to some people. If the university
  infrastructure allowed for reliable hosting of public keys, this
  task could be eased.
\item The server may serve fake code for some users which reveals more
  information. But in such a case, the hashes of the code received and
  the code expected will not match, and a concerned user can verify
  the same, or run a local copy of the code (which she has manually
  verified to be safe).
\end{itemize}

\section{Deployment}
The platform was well received, with more than 1800 students
registering on the website, and as many as 45 matched couples. There
were many people who had multiple matches, both among boys and girls.

Although girls are outnumbered on the campus $1:10$, among the
registrants, the ratio was close to $1:5$.

The backend was implemented in Golang, using the $iris$ web
framework. The frontend was written using Angular2 and TypeScript,
communicating with the backend using REST API of the backend. The
server resource usage was monitored and was observed to be very low
even during peak times, due to Golang's lightweight web
frameworks. MongoDB was used for data storage, and Redis was used to
ensure persistent sessions.

\begin{thebibliography}{}
\softraggedright
\bibitem[(2015)]{Senpai:1}
  Senpai: Solving the dating problem
\newblock SIGTBD 2017.

\end{thebibliography}

\end{document}

%%% Local Variables:
%%% mode: latex
%%% TeX-master: t
%%% End:
